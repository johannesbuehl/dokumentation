\chapter{Streamdeck}
	Das \textit{Elgato Streamdeck} ist eine programmierbare Makrotastatur mit 15 Displaytasten. Mithilfe der zugehörige Software von Elgato kann man eine maßgeschneiderte Makrotastatur erstellen, welche mit vielen Programmen direkt kommunizieren kann.

	Das verbaute Streamdeck wird allerdings mit der drittanbieter Software \textit{Bitfocus Companion} betrieben. Diese bietet ähnliche Funktionen, ist aber eher für Rundfunkanwendungen konzipiert. Es lassen sich ebenfalls viele Programme durch Netzwerkprotokolle direkt einbinden.

	Das Interface bildet 100 aufeinanderfolgende Seiten ab, zwischen denen auf- und abgeblättert oder umhergesprungen werden kann.

	Programmiert sind unter anderem Seiten, um:
	\begin{itemize}
		\item Programme zu starten
		\item Stream und Aufnahme zu starten und beenden
		\item Szenen in \Gls{OBS} zu wechseln
		\item PTZ-Presets abzurufen
		\item Einblendungen zu steuern
		\item die \Gls{PTZ-Kamera} fernzusteuern
		\item PTZ-Preset abzuspeichern
		\item Atem-Mischer fern zu steuern
	\end{itemize}

	\section{Einrichtung}
		Die komplette Einrichtung der Companion-Software erfolgt im Browser. Durch einen Klick auf das Symbol im Benachrichtigungsfelder der Taskleiste öffnet sich ein Fenster, mit welchem die Konfigurationsseite im Browser öffnen lässt (\textit{Launch GUI}).

		Dort kann nun sowohl die Belegung der einzelnen Buttons, als auch die Konfiguration der geladenen Modulen angepasst werden.

	\section{Konfiguration}
		\subsection{PTZ}
			Companion ermöglicht eine direkte Steuerung der \Gls{PTZ-Kamera} über das \textit{\Gls{Visca}-Protokoll}. Die Buttons, welche die verschiedenen PTZ-Presets abrufen, haben zwei Befehle hinterlegt:
			\begin{enumerate}
				\item PTZ-Preset abrufen: Dies sendet einen \Gls{Visca}-Befehl an die Kamera, um das entsprechende Preset anzufahren.
				\item Szenen-Wechsel in \Gls{OBS}: In \Gls{OBS} zu der Szene wechseln, welche dem Preset entspricht. Hierdurch wird die \Gls{stinger} ausgelöst.
			\end{enumerate}