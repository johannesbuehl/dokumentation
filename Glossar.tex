\newglossaryentry{Endstufe}{
	name={Endstufe},
	plural={Endstufen},
	description={
		Eine \textit{Endstufe} verstärkt ein Line-Signal von einem Mischpult auf ein höhere Leistung, sodass damit Lautsprecher angetrieben werden können
	}
	% ,see={}
}

\newglossaryentry{MADI}{
	name={MADI},
	% plural={},
	description={
		\textit{MADI}, als AES10 genormt, ist ein digitales Protokoll zur mehrkanaligen Audioübertragung welches häufig für digitale Stageboxen verwendet wird.
		Es wird für gewöhnlich über Ethernet- oder Glasfaserkabel, manchmal auch über Koaxialkabel übertragen.
		Allerdings ist MADI \underline{KEIN} Netzwerkprotokoll sondern eine Punkt-zu-Punkt-Verbindung.
		Es wird lediglich die hohen Datenraten von Netzwerkkabeln ausgenutzt.
		\\MADI wird von mehreren Herstellern verwendet; bei Soundcraft ist es das "`Standard Protokoll"'.
		\\MADI selbst ist zwar standardisiert, allerdings ist die Steuerung von Preamps oder Phantomspeißung proprietär und funktionert für gewöhnlich nur innerhalb des Ökosystem eines Herstellers
	}
	,see={digitale Stagebox,AES50,DANTE}
}

\newglossaryentry{digitale Stagebox}{
	name={digitale Stagebox},
	% plural={},
	description={
		Eine \textit{digitale Stagebox} erfüllt die gleiche Funktion wie eine "`reguläre'' Stagebox.
		Der Unterschied besteht darin, dass die Signale von der Stagebox zum Mischpult digital übertragen werden.
		Daher befinden sich in der Stagebox selbst bereits Vorverstärker und AD-Wandler.
		\\Der Vorteil besteht darin, dass nur ein einzelnes Netzwerk-, Glasfaser- oder Koaxialkabel zum Mischpult verlegt werden muss.
		Meist kann allerdings optional für Redundanzzwecke oder einen Signalsplit an ein zweites Mischpult ein zweites Kabel angeschlossen werden.
		Allerdings ist nun ein extra Stromanschluss für die Stagebox erforderlich
	}
	,see={Stagebox,AD-Wandler,Preamp}
}

\newglossaryentry{AD-Wandler}{
	name={AD-Wandler},
	% plural={},
	description={
		\textit{AD-Wandler} steht für \textit{Analog-Digital-Wandler} und ist ein Bauteil, welches analoge Signale (zum Beispiel von einem Mikrofon) in ein digitales Signal wandeln.
		Dadurch ist eine Verarbeitung des Signals in der digitalen Domäne möglich.
		\\Das Gegenstück wird dementsprechend \textit{DA-Wandler genannt}.
		\\In der (digitalen) Tontechnik befinden sich AD-Wandler in den Eingängen am Mischpult und digitalen Stageboxen.
		DA-Wandler können sich ebenfalls in Mischpult und Stagebox befinden.
		Es ist aber auch üblich, das Ausgangssignal digital an Lautsprechercontroller oder Endstufen zu schicken, da diese häufig für die weitere Signalverarbeitung sowieso digital arbeiten
	}
	,see={digitale Stagebox,Lautsprechercontroller}
}

\newglossaryentry{Lautsprechercontroller}{
	name={Lautsprechercontroller},
	% plural={},
	description={
		Ein Lautsprechercontroller ist ein einzelnes Gerät, welches ein (oder mehrere) Tonsignale Lautsprecherspezifisch bearbeitet.
		Die Funktionen können umfassen:
		\begin{multicols}{2}
			\begin{itemize}
				\item Verzögern der Signale, um Laufzeitverzögerungen des Schalls im Raum auszugleichen
				\item Equalizing der Signale, um Eigenheiten der Lautsprecher und/oder des Raumes auszugleichen
				\item Limitierung und Kompression der Signale
				\item automatische Feedback-Unterdrückung
			\end{itemize}
		\end{multicols}
	}
	,see={Feedback}
}

\newglossaryentry{Feedback}{
	name={Feedback},
	% plural={},
	description={
		\textit{Feedback}, auch \textit{Rückkopplung} genannt, bezeichnet die Rückführung eines Ausgangssignals auf den Eingang.
		In der Tontechnik ist mit Feedback eine ungewollte Signalschleife von Mikrofon und Lautsprecher gemeint.
		Dabei schwingt sich das Signal immer weiter auf bis es zu einem unangenehmen, lauten (und meist hohen) Pfeifton kommt.
		\\Es gibt auch ein "`Feedback-Light"', bei welchem sich die Rückkopplung nicht aufschwingt aber mitklingt und somit die Verständlichkeit stört.
		Hierbei ist das erneut vom Mikrofon aufgenommene Signal leiser als das ursprüngliche Signal, aber dennoch laut genug, um eine hörbare Kopplung zu erzeugen.
	}
	% ,see={}
}

\newglossaryentry{Preamp}{
	name={Preamp},
	% plural={},
	description={
		Ein \textit{Preamp}, auf deutsch \textit{Vorverstärker} ist eine elektronische Schaltung, welche ein Signal mit niedriger Spannung (= niedrige Lautstärke) auf eine höhere Spannung verstärkt.
		Diese Verstärkung wird in Dezibel (dB) angegeben.
		Hierbei wird nicht nur das Signal, sonder zwangsweise immer auch Hintergrundgeräusche und -rauschen mitverstärkt.
		Zusätzlich zu dem Signalrauschen haben einige Preamps bei hohen Verstärkungen ein hörbares Eigenrauschen.
		\\In der Tontechnik werden Preamps benötigt, um die niedrigen Ausgangspegel von Mikrofonen und Instrumenten auf ein Level zu bringen, das eine Weiterverarbeitung ermöglicht.
	}
	% ,see={}
}

\newglossaryentry{Stagebox}{
	name={Stagebox},
	% plural={},
	description={
		Eine \textit{Stagebox} ist ein Verteilerkasten auf der Bühne (englisch \textit{Stage}) für Mikrofone.
		Diese bündeln mehrere Signalleitungen und verbinden sie mit dem Mischpult.
		Dabei können sowohl Eingänge (für Mikrofone etc.) als auch Ausgänge (für Monitore etc.) in einer Stagebox vorhanden sein.
		Es gibt sowohl fest verbaute als auch mobile Versionen.
		\\Es gibt sowohl analoge als auch digitale Stageboxen, wobei die analogen lediglich ein Verlängerungskabel sind
	}
	,see={digitale Stagebox}
}

\newglossaryentry{AES50}{
	name={AES50},
	% plural={},
	description={
		\textit{AES50} ist ein digitales Protokoll zur mehrkanaligen Audioübertragung ähnlich zu MADI.
		Allerdings sind beide Protokolle \underline{NICHT} kompatibel.
		\\AES50 wird hauptsächlich von Geräten aus der Unternehmensgruppe Music Tribe, zu der unter anderem Behringer und Midas gehören, verwendet
	}
	,see={MADI}
}

\newglossaryentry{DANTE}{
	name={DANTE},
	% plural={},
	description={
		\textit{DANTE} ist ein digitales Protokoll zur mehrkanaligen Audioübertragung.
		Im Gegensatz zu MADI oder AES50 ist es ein Netzwerkprotokoll.
		Das heißt, es kann über normale Netzwerktechnik geroutet werden.
		\\Dante wird häufig bei größeren Installationen und Produktionen herstellerübergreifend eingesetzt.
		Genauso wie bei MADI sind die Steuerung von Preamps und der Phantomspeißung proprietär und für funktioneren für gewöhhnlich nur innerhalb des Öko-Systems
	}
	,see={MADI,AES50}
}

\newglossaryentry{Patch}{
	name={Patch},
	plural={Patches},
	description={
		Bezeichnet eine Steckverbindung im Signalpfad.
		Relevant ist hierbei der Datenfluss, von welchem Element zu welchem Element das Signal fließt.
		Im analogen Bereich ist das in der Regel eine Kabelverbindung; in der digitalen Domäne können Signale in der Regel durch Zuweisungen gepatcht werden.
	}
}

\newglossaryentry{Monitor}{
	name={Monitor},
	plural={Monitore},
	description={
		Ein Monitor bezeichnet in der Tontechnik einen Lautsprecher, der zu den Akteuren (meist Musiker) gerichtet ist, damit sie sich selbst hören können.
		In der Regel liegt auf den Monitoren eine andere Abmischung als auf den Publikumslautsprechern.
		Mit unter hat sogar jeder Musiker einen eigenen Monitor mit einem eigenen Mix
	}
	% ,see={}
}

\newglossaryentry{Routing}{
	name={Routing},
	description={
		Bezeichnet den Fluss von Signalen
	}
	,see={Patch}
}

\newglossaryentry{Multicore}{
	name={Multicore},
	description={
		Ein Kabel, in dem mehrere einzelne Kanäle zusammengefasst werden.
		Dadurch kann eine große Anzahl an Signalen durch ein einzelnes Kabel übertragen werden.
	}
	% ,see={},
}

\newglossaryentry{digitales Multicore}{
	name={digitales Multicore},
	% plural={},
	description={
		Ein \textit{digitales Multicore} erfüllt die gleiche Funktion wie ein analoges Multicore.
		Allerdings werden die Signale in der Stagebox direkt digitalisiert und als digitales Signal an das Mischpult gesendet.
		Hierdurch können (vor allem bei hohen Kanalzahlen) kleinere Kabel verwendet werden (meist Ethernet- oder Glasfaserkabel).
	}
	,see={Multicore,digitale Stagebox}
}

\newglossaryentry{Lautheit}{
	name={Lautheit},
	description={
		\textit{Lautheit} ist eine Größe welche die von Menschen empfundene Lautstärke eines Signals abbilden soll.
		Damit unterscheidet sie sich vom Peak-Wert, welcher den Maximalwert eines Signals angibt.
		Pegelanzeigen zeigen in der Regeln den \Gls{Peak}-Wert an und eignen sich damit nur bedingt für eine Aussage über die Lautheit.\\
		Die Lautheit wird meistens in \textit{LU} (\textit{\textbf{L}oudness \textbf{U}nits}) angegeben, häufig auch in Bezug auf den maximal möglichen Wert (\textit{\textbf{F}ull \textbf{S}cale}-Wert): \textit{LUFS}
	},
	see={Peak}
}

\newglossaryentry{Peak}{
	name={Peak},
	description={
		Der Peak-Wert bezeichnet den Spitzenwert eines (Audio)-Signals.
		Mit dem Peak-Wert lassen sich nur bedingt Aussagen über die Lautheit machen
	}
	,see={TruePeak,Lautheit}
}

\newglossaryentry{TruePeak}{
	name={True-Peak},
	description={
		True Peaks, auch Intersample Peaks genannt, sind Peaks, die bei einer Umwandlung ins Analoge oder in einen anderen (verlustbehafteten) Audiocodec auftreten können.
		Obwohl die einzelnen Samples eines Signals nicht lauter als \SI{0}{dBFS} werden können, kann der Pegel zwischen den Samples rechnerisch noch weiter ansteigen.
		Diese Übersteuerungen können dann auftreten, wenn das Signal zurück in ein analoges Signal oder in einen verlustbehafteten Codec umgewandelt wird.
		Manche Peak-Anzeigen haben einen True Peak-Modus, welcher diese berechnet und anzeigt.\\
		Um solche True Peaks zu vermeiden, wird meist ein zusätzlicher Headroom von etwa \SI{1}{dBFS} gelassen
	},
	see={Peak,Headroom}
}

\newglossaryentry{Headroom}{
	name={Headroom},
	description={
		Der \textit{Headroom} bezeichnet eine Aussteuerungsreserve, welche in einem Signal übrig ist.
		Es ist der Unterschied zwischen dem maximalen und dem maximal möglichen Pegel
	}
	,see={TruePeak}
}

\newglossaryentry{VST-Plugin}{
	name={VST-Plugin},
	description={
		VST steht für Virtual Studio Technology und ist eine Programmierschnittstelle für Audio-Plugins
	}
}

\newglossaryentry{SDI}{
	name={SDI},
	description={
		SDI, kurz für Serial Digital Interface, ist eine serielle Übertragungsschnittstelle für digitale Videosignale über Koaxialkabel oder Lichtwellenleiter.
		Es ermöglicht im Gegensatz zu HDMI Kabellängen von bis zu 100 Metern (über Koaxialkabel)
	}
}

\newglossaryentry{NDI}{
	name={NDI},
	description={
		NDI, kurz für Network Device Interface, ist eine Spezifikationen zur Übertragung digitaler Videosignale über ein Computernetzwerk
	}
}

\newglossaryentry{PTZ-Kamera}{
	name={PTZ-Kamera},
	description={
		Eine PTZ-Kamera ist eine Kamera, deren \textbf{P}an, \textbf{T}ilt und \textbf{Z}oom ferngesteuert werden kann
	}
}

\newglossaryentry{FoH_full}{
	name={Front of House},
	description={
		\textit{Front of House} bezeichnet den Ort, an dem sich bei einer Veranstaltung die Ton-, Video- und Lichttechnik befindet
	}
}

\newglossaryentry{FoH}{
	name={FoH},
	description={
		Abkürzung für \Gls{FoH_full}
	},
	see={FoH_full}
}

\newglossaryentry{PoE_full}{
	name={Power over Ethernet},
	description={
		Power over Ethernet bezeichnet das Verfahren, elektronische Geräte über das Ethernet-Kabel mit Strom zu versorgen.
		Dies ermöglicht den Anschluss eines Gerätes ohne zusätzliche Stromversorgung.
		Dies wird zum Beispiel häufig für WLAN-Access-Points verwendet
	}
}

\newglossaryentry{PoE}{
	name={PoE},
	description={
		Abkürzung für \Gls{PoE_full}
	},
	see={PoE_full}
}

\newglossaryentry{OBS} {
	name={OBS},
	description={
		Open Broadcaster Software ist eine Videomischsoftware, mit welcher Bild-, Video- und Audiosignale live zusammengesetzt und gemischt werden können
	}
}

\newglossaryentry{Visca} {
	name={Visca},
	description={
		Das Visca-Protokoll ist ein von Sony entwickeltets Protokoll, zur Kommunikation mit Videokameras
	}
}

\newglossaryentry{stinger} {
	name={Stinger-Transition},
	description={
		Stinger-Transitions sind eine Möglichkeit in \Gls{OBS}, einen Übergang zwischen zwei Szenen zu gestalten.
		Hierbei wird eine Videodatei über das eigentliche Bild gelegt und nach einer vorher definierten Zeit zur neuen Szene gewechselt.
		Durch transparente Anteile in der Videodatei sind weiche Übergänge möglich
	},
	see={OBS}
}

\setglossarystyle{altlist}
\printglossary