\chapter{Grundlagen - Videotechnik}
	\section{Benutzung}
		Um einen normalen Gottesdienst zu Streamen, ist wenig Vorbereitung und Aufbau notwendig
		\subsection{Aufbau \& Vorbereitung}
			\subsubsection*{Aufschließen}
				In dem hohen, abgeschlossenen Rack befindet sich der Streamingrechner.
				Es muss sowohl der Deckel als auch die vordere Abdeckung abgenommen werden.
			\subsubsection*{Lampen}
				Für eine gute Ausleuchtung werden alle Lampen aus \tableref{table:grundlagen:video:aufbau:lampen} benötigt.

				\textbf{WICHTIG:} Aufgrund der alten Stromversorgung ist es zwingend notwendig, die beiden Stufenlinsenstecker mit einem Abstand von mindestens $\varDelta t_{min} = \SI{5}{\min}$ einzustecken.

				\begin{table}[H]
					\caption{Lampenschaltorte}
					\label{table:grundlagen:video:aufbau:lampen}
					\centering

					\begin{tabular}{ll}
						\toprule
						\textbf{Lampe} & \textbf{Schaltort} \\
						\midrule
						Kirchenbeleuchtung: Stufe 1 & Wandschalter am Eingang \\
						Kirchenbeleuchtung: Stufe 2 & Kontrollkasten am Eingang \\
						Altarraumbeleuchtung: Taufstein & Kontrollkasten am Eingang \\
						Altarraumbeleuchtung: Kanzel & Kontrollkasten am Eingang\\
						Stufenlinsen 1 & Stecker hinter Flügel \\
						Stufenlinsen 2 & Stecker vor Sakristeieingang \\
						\bottomrule
					\end{tabular}
				\end{table}
			\subsubsection*{Peripherie}
				In der Sakristei befindet sich ein Holzbrett auf dem sich die Peripherie befindet.
				Dieses wird auf den Tisch neben dem Mischpult so nahe wie möglich an das Rack mit dem Computer gestellt.
				Unter die linke, hintere Ecke empfiehlt es sich, zwei blaue Gesangsbücher (\textit{"`Wo wir dich loben, wachsen neue Lieder plus"'}) zu unterlegen.

				Das Brett wird gemäß der \tableref{table:grundlagen:video:aufbau:peripherie} angeschlossen.

				Nachdem alle Kabel angeschlossen sind, kann der Rechner gestartet werden.

				\begin{table}[h]
					\caption{Verkabelung des Brettes mit dem Rack}
					\label{table:grundlagen:video:aufbau:peripherie}
					\centering

					\begin{tabular}{ll}
						\toprule
						\textbf{Kabel} & \textbf{Ort} \\
						\midrule
						VGA & Grafikkarte an der Rechner-Rückseite \\
						HDMI & Grafikkarte an der Rechner-Rückseite (neben VGA) \\
						USB & USB 3.0 Anschluss an der Rechner-Rückseite (blaue Buchse) \\
						Strom & weiße Steckdosenleiste an der oberen Rackkante \\
						\bottomrule
					\end{tabular}
				\end{table}
			\subsubsection*{Programme}
				Nach dem Rechnerstart können mit dem Button 10.2 (Seite \textit{Streaming Setup}, oberste Reihe, 1. Zeile, 2. Spalte) alle Programme gleichzeitig gestartet werden.
				Es öffnen sich: \textit{OBS}, \textit{PPT NDI} und \textit{Firefox} mit der \textit{Livestream-Studio Seite}.