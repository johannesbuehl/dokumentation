\chapter{Grundlagen - Videotechnik}
	\section{Benutzung}
		Um einen normalen Gottesdienst zu Streamen, ist wenig Vorbereitung und Aufbau notwendig
		\subsection{Rack Aufschließen}
			In dem hohen, abgeschlossenen Rack befindet sich der Streamingrechner.
			Es muss sowohl der Deckel als auch die vordere Abdeckung abgenommen werden.
		\subsection{Lampen anschalten}\label{ssec:grundlagen:benutzung:lampen}
			Für eine gute Ausleuchtung werden alle Lampen aus \tableref{table:grundlagen:video:aufbau:lampen} benötigt.

			\textbf{WICHTIG:} Aufgrund der alten Stromversorgung ist es zwingend notwendig, die beiden \Glsplural{Stufenlinse}-Stecker mit einem Abstand von mindestens $\varDelta t_{min} = \SI{5}{\min}$ einzustecken.

			\begin{table}[H]
				\caption{Lampenschaltorte}
				\label{table:grundlagen:video:aufbau:lampen}
				\centering

				\begin{tabular}{ll}
					\toprule
					\textbf{Lampe} & \textbf{Schaltort} \\
					\midrule
					Kirchenbeleuchtung: Stufe 1 & Wandschalter am Eingang \\
					Kirchenbeleuchtung: Stufe 2 & Kontrollkasten am Eingang \\
					Altarraumbeleuchtung: Taufstein & Kontrollkasten am Eingang \\
					Altarraumbeleuchtung: Kanzel & Kontrollkasten am Eingang\\
					\Glsplural{Stufenlinse} 1 & Stecker hinter Flügel \\
					\Glsplural{Stufenlinse} 2 & Stecker vor Sakristeieingang \\
					\bottomrule
				\end{tabular}
			\end{table}
		\subsection{Peripherie anschließen}\label{ssec:grundlagen:benutzung:peripherie}
			In der Sakristei befindet sich ein Holzbrett auf dem sich die Peripherie befindet.
			Dieses wird auf den Tisch neben dem Mischpult so nahe wie möglich an das Rack mit dem Computer gestellt.
			Unter die linke, hintere Ecke empfiehlt es sich, zwei blaue Gesangsbücher (\textit{"`Wo wir dich loben, wachsen neue Lieder plus"'}) zu unterlegen.

			Das Brett wird gemäß der \tableref{table:grundlagen:video:aufbau:peripherie} angeschlossen.

			Nachdem alle Kabel angeschlossen sind, kann der Rechner gestartet werden.
			(Wird der Rechner gestartet, bevor die komplette Peripherie verbunden ist, verbindet sich \textit{Companion} nicht mit dem Streamdeck)

			\begin{table}[h]
				\caption{Verkabelung des Brettes mit dem Rack}
				\label{table:grundlagen:video:aufbau:peripherie}
				\centering

				\begin{tabular}{ll}
					\toprule
					\textbf{Kabel} & \textbf{Ort} \\
					\midrule
					VGA & Grafikkarte an der Rechner-Rückseite \\
					HDMI & Grafikkarte an der Rechner-Rückseite (neben VGA) \\
					USB & USB 3.0 Anschluss an der Rechner-Rückseite (blaue Buchse) \\
					Strom & weiße Steckdosenleiste an der oberen Rackkante \\
					\bottomrule
				\end{tabular}
			\end{table}
		\subsection{Programme starten und vobereiten}
			Nach dem Rechnerstart können mit dem Streamdeck-Button 10.2 (Seite \textit{Streaming Setup}, 1. Zeile, 2. Spalte) alle Programme gleichzeitig gestartet werden.
			Es öffnen sich: \textit{OBS}, \textit{PPT NDI} und \textit{Firefox} mit der \textit{Livestream-Studio-Seite}.
			\subsubsection{PPT NDI}
				In \textit{PPT NDI} muss die aktuelle PowerPoint ausgewählt werden.
				Hierzu auf den \textit{Open}-Button klicken (\figref{fig:ppt-ndi:interface}) und dort die passende PowerPoint-Präsentation auswählen (\figref{fig:ppt-ndi:open-dialog}).
			\subsubsection{OBS}
				In \Gls{OBS} muss der Livestream ausgewählt werden, damit YouTube weiß, für welchen Livestream das Video ist.
				Die Funktion wird über den \textit{Manage Broadcast}-Button aufgerufen (\figref{fig:obs:interface:stream-select}).
				Im sich nun öffnenden Fenster wechselt man zum Tab \textit{Select Existing Broadcast}, um einen zuvor erstellten Livestream auszuwählen (\figref{fig:obs:stream-select}).
				In der Liste wird nun der korrekte Livestream ausgewählt und über den Button \textit{Select Broadcast} bestätigt (\figref{fig:obs:stream-select-list}).\\

				\textbf{WICHTIG:} \underline{NICHT} den Button \textit{Select broadcast and start streaming} drücken, da \Gls{OBS} ansonsten direkt zu streamen beginnt und somit der Stream auf YouTube direkt live ist.
			\subsubsection{Firefox / YouTube}
				In Firefox ist bereits das YouTube-Livestream-Studio geöffnet, in welchem ebenfalls eine Auflistung der geplanten Livestreams zu sehen ist.
				Aus dieser Liste wird der korrekte Livestream ausgewählt (\figref{fig:youtube:livestream-studio}) und man gelangt zu einer Übersichtseite des Livestreams, auf welcher Einstellungen verändert werden können oder man den Livechat sehen und moderieren kann.

				\textbf{Hinweis:} Dieser Schritt ist nicht zwingend für das funktioneren des Livestreams notwendig, sondern gibt lediglich eine bessere Übersicht.
		\subsection{Livestream starten}
			Um den Livestream zu starten, gibt es zwei Möglichkeiten.
			In beiden Fällen wird gleichzeitig automatisch auch eine lokale Aufnahme gestartet.
			\subsubsection{OBS}
				In \Gls{OBS} kann der Button \textit{Start Streaming} (\figref{fig:obs:interface:stream-start}) verwendet werden, um Live zu gehen.
			\subsubsection{Streamdeck}
				Alternativ kann der Livestream auch über den Streamdeck-Button 10.4 \textit{Stream staten} (Seite \textit{Streaming Setup}, 1. Zeile, 4. Spalte) gestartet werden.
		\subsection{Livestream stoppen}
			Am Ende des Gottesdientes muss der Livestream wieder gestoppt werden - für gewöhnlich nach dem Ende des Ausgangsstückes des Organisten / der Band.
			Hierfür gibt es analog zum Start des Livestreams zwei Möglichkeiten.
			Bei beiden Möglichkeiten wird die Aufnahme \textbf{NICHT} gestoppt, sondern läuft weiter.
			Diese kann entweder manuell gestoppt werden oder beim Schließen von \Gls{OBS} wird man gewarnt, dass mit dem Schließen auch die Aufnahem beendet wird.
			\subsubsection{OBS}
				Ist ein Livestream aktiviert, wird der ehemalige \textit{Start Streaming}-Button zu einem \textit{Stop Streaming}-Button (\figref{fig:obs:interface:stream-stop}), über welchen der Stream von OBS wieder beendet werden kann.
			\subsubsection{Streamdeck}
				Alternativ kann der Livestream auch wieder über den Streamdeck-Button 10.4 \textit{Stream starten} (Seite \textit{Streaming Setup}, 1. Zeile, 4. Spalte) gestoppt werden.
		\subsection{Ausschalten \& Abbau}
			Beim Ausschalten muss nichts spezielles beachtet werden - einfach nur Windows herunterfahren.
			
			Beim Abbau müssen die Kabel aus \secref{ssec:grundlagen:benutzung:peripherie} wieder abgesteckt und auf das Brett gelegt werden.
			Ebenfalls muss die Peripherie ordentlich auf dem Brett verräumt sein, damit nichts herunterfällt oder übersteht.
			Dann wird das Brett wieder zurück in die Sakristei gestellt.

			Die Lampen aus \secref{ssec:grundlagen:benutzung:lampen} müssen wieder ausgeschalten, beziehungsweise ausgesteckt werden.
			Beim Ausstecken muss bei den Stufenlinsen keine Wartezeit eingehalten werden.

			Ist alles verräumt und verstaut, muss das Rack wieder abgeschlossen werden.