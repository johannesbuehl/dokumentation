\chapter{Kameras}
	\section{PTZ}
		\subsection{Steuerung}
			Die \Gls{PTZ-Kamera} wird über eine Netzwerkanbindung ferngesteuert.
			Es wäre auch eine Steuerung über RS-232 oder RS-485 möglich, was beispielsweise von dedizierten Controlpanels genutzt wird.
			Die installierte \Gls{PTZ-Kamera} wird allerdings von \refname{companion} angesteuert, welches über das Netzwerk kommuniziert.

			Die Kamera unterstützt mehrere Steuerprotokolle, wobei hier das \Gls{Visca}-Protokoll eingesetzt wird.
		\subsection{Videosignal}
		\subsection{Stromversorgung}
			Die \Gls{PTZ-Kamera} wird über \Gls{PoE} mit Strom versorgt.
			Da die Kamera für die Steuerung sowieso an das Netzwerk angeschlossen ist, kann somit ein Kabel eingespart werden.
		\subsection{Parkposition}
			Damit sich während der Nichtbenutzung der Kamera möglichst wenig Staub auf der Linse ablagert, fährt die Kamera beim Schließen von OBS eine Parkposition an, in welcher sie senkrecht nach unten schaut.

			Hierzu ist im \Gls{OBS}-Plugin \textit{Advanced Scene Switcher} ein Makro hinterlegt, welches beim Schließen von OBS ausgeführt wird. Es wird eine Exe ausgeführt, welche einen \Gls{Visca}-Befehl an die \Gls{PTZ-Kamera} sendet.

			Das Programm ist ein Python-Skript, welches als Standalone-Exe kompiliert wurde. Über das Modul \textit{visca-over-ip} kann mit \Glsplural{PTZ-Kamera} kommuniziert werden. Vor dem erstellen der Exe-Datei muss die IP-Adresse angepasst werden.
	\section{Sony A6000}