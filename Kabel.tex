\chapter{Verkabelung \& Routing}
	\section{SDI}
		Alle Videostrecken wurden mit \Gls{SDI}-Koaxial-Kabeln durchgeführt, da die einige der Strecken zu lang für eine zuverlässige HDMI-Verbindung sind.
		Im Computer ist hierfür eine \textit{Blackmagic Design DeckLink Duo 2} verbaut.
		Diese bietet 4 Bidirektionale 3G-\Gls{SDI}-Anschlüsse und einen Sync-Eingang; dieser wird allerdings nicht benutzt.

		Es sind folgende Kabelstrecken installiert:
		\begin{table}[h]
			\caption{Belegung der \Gls{SDI}-Capture-Karte \textit{Blackmagic Design DeckLink Duo 2}}
			\centering
			
			\begin{tabular}{ccl}
				\toprule
				Anschluss & Konfiguration & Bezeichnung \\
				\midrule
				Sync & - & nicht verwendet \\
				1 & Eingang & \Gls{PTZ-Kamera} \\
				2 & Eingang & Leitung Altarraum \\
				3 & Eingang & Patchfeld \Gls{FoH} \\
				4 & Ausgang & Beamer Gemeindesaal \\
				\bottomrule
			\end{tabular}
		\end{table}

		Die PTZ-Kamera ist fest angeschlossen, ebenso der Beamer.
		Die Leitung in den Altarraum endet hinter dem Leimbinder am Flügel wo noch einige restliche Meter Kabel aufgewickelt sind.
		Der dritte Anschluss ist an das Patchfeld im Rack angeschlossen.
		Diese beiden freien Leitungen sind für weitere Videoquellen gedacht, die bedarfsorientiert genutzt werden können.
		Hierzu gibt es auch noch ein weiteres, loses \Gls{SDI}-Kabel und Verbinder um bestehende Kabel zu verlängern.

	\section{Netzwerk}
		Das \Gls{FoH} ist über einen einzelne Gigabit Leitung an das Netzwerk und somit auch an das Internet angeschlossen.
		Der Haupt-Switch befindet sich unter der Treppe im Kindergarten.
		Von hieraus liegen 4 Leitungen auf die Bühne in die Theke des Jugendcafes.
		An einen dieser Ports ist ein weiteres Kabel angeschlossen, welches auf den Speicher und dann entlang der wandseitigen Lampen im Gemeindesaal führt bis in das Rack.
		Dort befindet sich ein 8-Fach Switch, welcher die verschiedenen Geräten verbindet.

		Da der Haupt-Switch \Gls{PoE} zur Verfügung stellt, ist keine zusätzliche Stromversorgung des Technik-Switches notwendig.
		Ebenfalls wird innerhalb der beiden vierer Gruppen \Gls{PoE} weitergeleitet.
		Ebenfalls wird die \Gls{PTZ-Kamera} über \Gls{PoE} mit Strom versorgt.

		Desweiteren ist der Streamingrechner an den Switch angeschlossen.

		\subsection{Netzwerkkabel in Altarraum}
			Es führt ein weiteres, ungenutztes Netzwerkkabel in den Altarraum.
			Sie endet hinter dem Leimbinder neben dem Flügel, wo auch der Rest des Kabels aufgerollt ist.
			Diese Leitung ist für ein digitales \Gls{Multicore} gedacht (Siehe auch \nameref{kabel:ton:digitales_multicore}).
			
			Die meisten digitalen \Gls{Multicore}-Protokolle (z.B. \Gls{MADI}, \Gls{AES50}, SLink) sind jedoch \underline{KEINE} Netzwerkprotokolle sondern Punkt-zu-Punkt-Verbindungen welche nur die hohen Datenraten von Netwerkkabeln ausnutzen. (Ausnahme: \Gls{DANTE}, AVB)
	\section{Tontechnik}
		Das \Gls{Routing} findet sowohl im analogen als auch im digitalen Bereich statt.
		\subsection{Kirche}
			Die Verkabelung in der Kirche kann in mehrere Teile unterteilt werden:
			\subsubsection{Multicore von Altarraum zu Mischpult}
				Vom \Gls{FoH} aus führt ein 16-kanäliges \Gls{Multicore} in den Altarraum.
				Davon sind die ersten 10 Kanäle als Eingänge und die restlichen 4 als Ausgänge ausgeführt.

				Es wird genutzt für:
				\begin{itemize}
					\item Altarmikrofon
					\item Kanzelmikrofon
					\item Raummikrofone
					\item \nameref{kabel:ton:multicore_altarraum}
				\end{itemize}
			\subsubsection{Multicore im Altarrum}\label{kabel:ton:multicore_altarraum}
				Im Altarraum liegt ein 12 kanäliges Multicore mit einer Kabeltrommel.
				Davon sind die ersten 10 Kanäle als Eingänge und die restlichen 2 als Ausgänge ausgeführt.

				Es wird genutzt für:
				\begin{itemize}
					\item Flügel-Mikrofone
					\item \glsplural{Monitor}
				\end{itemize}

				Die freien Kanäle können für weitere, nicht dauerhaft aufgebaute Mikrofone verwendet werden (z.B. Gesang oder Instrumente).
				Sie sind bis ins Mischpult gepatcht und sind auf mehreren Kanälen bereits eingerichtet.
			\subsubsection{Multicore von Rack in Metallschrank}
				Ein weiteres, kurzes Multicore verläuft zwischen dem Rack und dem Metallschrank, in welchem sich die Endstufen befinden.
				Es verbindet den \Gls{Lautsprechercontroller} mit den drei \Glsplural{Endstufe} und verlängert die Kabel von der Empore in das Rack.
			\subsubsection{Empore in Metallschrank}
				Von der Empore führen zwei Mikrofonkable in den Metallschrank.
				Auf dem ersten Kanal befindet sich das Orgelmikrofon, der zweite ist unbelegt und endet auf der Rückseite des Lichtschalters am Leimbinder neben der Orgel.
			\subsubsection{Digitales Multicore in Altarraum}\label{kabel:ton:digitales_multicore}
				Es liegt ein unbenutztes Netzwerkkabel vom \Gls{FoH} zum Leimbinder neben dem Flügel.
				Gedacht ist es für ein \Gls{digitales Multicore}, welches für vereinzelte Veranstaltungen von Extern mitgebracht wird oder einen eventuellen, zukünftigen Kauf.
		\subsection{Rack}
				Zwischen Rack und Mischpult und auch innerhalb des Mischpults sind weitere Verkabelungen.
				Die Verbindungen zum Mischpult sind mit Einzelkabeln ausgeführt und sind:
				\begin{itemize}
					\item 3 Funkempfänger
					\item CD-Spieler (Stereo)
					\item Lautsprecher-Controller (Stereo)
					\item USB zum Streaming-Rechner
					\item Stromversorgung des Mischpults
				\end{itemize}
		\subsection{Digital}
			Im Mischpult kann das Routing auf mehrere Arten vom linearen Signalfluss eines analogen Pultes abweichen.
			Es gibt folgende, individuelle Ebenen im Routing, die beliebig zur nähsten Ebene zugewiesen werden kann.
			\begin{enumerate}
				\item Signaleingänge in das Mischpult (Mischpult, Stagebox, USB, ...)
				\item Kanäle
				\item Fader (auf den verschiedenen Ebenen)
			\end{enumerate}
			Das heißt Beispielsweise, dass XLR-Eingang 15 des Mischpults auf Kanal 20 geroutet wird.
			Kanal 20 wiederrum liegt wiederrum auf dem Fader C3.
			\\Hierbei können mehrere Kanäle den selben Eingang benutzen.
			Da die komplette Signalverarbeitung digital stattfindet (mit Ausnahme des Preamps und der Phantomspeißung natürlich), können für beide Kanäle unterschiedliche Einstellungen festgelegt werden.
			\\Ebenso kann ein Kanal mehreren Fader zugewiesen werden.
			Diese reagieren dann komplett synchron, auch über mehrere Ebenen hinweg.