\chapter{Software}
	\section{OBS}
		Über \Gls{OBS} werden die verschiedenen Videosignale gemischt, aufgenommen und an YouTube geschickt.

		\subsection{Grundfunktionen}
			In \Gls{OBS} werden verschiedene Bild-Kompositionen in sogenannten Szenen gespeichert.
			In einer Szene kann eine vielzahl von verschiedenen Quellen (Sources) verwendet werden, unter anderem:
			\begin{itemize}
				\item reine Audioeingänge
				\item Bildschirmaufnahmen
				\item Fensteraufnahmen
				\item Bilder
				\item Medienquellen (Video- oder Musikdateien)
				\item NDI (durch \nameref{obs:plugins:ndi})
				\item andere Szenen
				\item Text
				\item Webcams
				\item Blackmagic Decklink Videoeingänge
			\end{itemize}
			Indem man sich in \Gls{OBS} mit den YouTube-Zugangsdaten anmeldet, kann man in \Gls{OBS} entweder Livestreams erstellen oder einen bereits geplanten Livestream auswählen und diesen streamen.
		\subsection{Plugins}
			\Gls{OBS} kann durch verschiedene Plugins in seiner Funktion erweitert werden.
			Installiert sind:
			\begin{itemize}
				\item obs-websocket
				\item Advanced Scene Switcher
				\item obs-ndi
				\item Audio-Monitor
			\end{itemize}

			\subsubsection{obs-websocket}
				\textit{obs-websocket} ermöglicht die Fernsteuerung von OBS durch einen Websocket.
				Dieser wird zum Beispiel von Companion genutzt, um Szenen zu wechseln oder Informationen über den aktuellen Zustand von OBS zu erlangen.
			\subsubsection{Advanced Scene Switcher}\label{obs:plugins:ass}
				Mit \textit{Advanced Scene Switcher} können Abläufe in \Gls{OBS} automatisiert werden.
				Beispielsweise kann am Ende einer Medienquelle automatisch die Szene gewechselt werden oder Programme beim Schließen von \Gls{OBS} ausgeführt werden.
			\subsubsection{obs-ndi}\label{obs:plugins:ndi}
				\textit{obs-ndi} ermöglicht die Nutzung von \Gls{NDI}-Signalen in \Gls{OBS}.
				Es fügt eine \Gls{NDI}-Source hinzu und ermöglicht eine \Gls{NDI}-Ausgabe einzelner Szenen oder des Programms.
			\subsubsection{Audio-Monitor}
				\textit{Audio-Monitor} fügt genau steuerbare Audioabhörmöglichkeiten hinzu.
				Die Standardmöglichkeiten von OBS sind in einem Untermenü versteckt und müssen erneut aktiviert werden, wenn der Kopfhörer neu eingesteckt wurde.

		\subsection{Konfiguration}
			\subsubsection{Intro}
				Der Ablauf des Intros wird durch \nameref{obs:plugins:ass}-Makros automatisiert.
				Es muss lediglich wenn die Kirchenglocken ausgeschalten werden zu der Szene \textit{Intro} gewechselt werden.

				Mit dem Start des Streams wird zur Szene \textit{Glocken} gewechselt.
				Diese zeigt die Powerpoint, spielt ein Glockenläuten ab und mutet den Mischpultton.
				Mit dem Ausschalten der (echten) Glocken wird auf die Szene \textit{Intro} gewechselt.
				Dadurch wird das Intro-Video abgespielt und spielt ein ausklingendes Glockenläuten ab.
				Nach dessen Ende wird wieder zur PowerPoint zurückgeschalten und der Mischpultton wieder entmutet.
				
				\begin{table}[H]
					\caption{Die Szenen für die Introsequenz}
					\centering

					\begin{tabular}{lll}
						\toprule
						\multicolumn{1}{c}{Szene} & Quelle & Quellentyp \\
						\midrule
						\textit{Glocken} & \parbox{0.3\textwidth}{PowerPoint\\Glockenläuten-Start.wav\\Glockenläuten-Loop.wav} & \parbox{0.2\textwidth}{Szene\\Medienquelle\\Medienquelle} \\ \midrule
						\textit{Intro} & \parbox{0.3\textwidth}{Intro.mp4\\Glockenläuten-Ende.wav} & \parbox{0.2\textwidth}{Medienquelle\\Medienquelle} \\
						\bottomrule
					\end{tabular}
				\end{table}

				\begin{table}[H]
					\caption{Die Makros für die Intro-Automatisierung}
					\centering

					\begin{tabular}{lll}
						\toprule
						\multicolumn{1}{c}{Name} & \multicolumn{1}{c}{Bedingung} & \multicolumn{1}{c}{Aktionen} \\
						\midrule
						\textit{on\_stream\_start} & Streaming gestartet & \parbox{0.5\textwidth}{
								Zu Szene \texttt{Glocken} wechseln\\
								Medienquelle \texttt{Glocken\-läuten-Start.wav} starten\\
								Mikrofon (Mischpult-Ton) muten
						} \\ \midrule
						\textit{glocke\_intro\_to\_loop} & \parbox{0.25\textwidth}{\texttt{Glocken\-läuten-Start.wav} ist $t_R = \SI{0,3}{\second}$ vor Ende} & \parbox{0.5\textwidth}{Medienquelle \texttt{Glocken\-läuten-Loop.wav} starten} \\ \midrule
						\textit{stop\_glocke\_on\_intro} & Aktive Szene ist \texttt{Intro} & \parbox{0.5\textwidth}{Medienquelle \texttt{Glocken\-läuten-Loop.wav} stoppen} \\ \midrule
						\textit{switch\_to\_PP\_after\_intro} & \parbox{0.25\textwidth}{Medienquelle \texttt{Intro.mp4} ist zu Ende} & \parbox{0.5\textwidth}{
							Zu Szene \texttt{PowerPoint} wechseln\\
							Mikrofon (Mischpult-Ton) entmuten
						} \\
						\bottomrule
					\end{tabular}
				\end{table}
			\subsubsection{PTZ-Kamera parken}
				Mit einem \nameref{obs:plugins:ass}-Makro wird die \Gls{PTZ-Kamera} beim Schließen von OBS in eine Park-Position gefahren.
				Siehe hierzu auch \nameref{cam:ptz:park}.

				\section{PPT NDI}
		Alle Text-Einblendungen werden mit PowerPoint erstellt.
		Eine Vorlage befindet sich im Google-Drive Verzeichnis.

		PPT NDI läd eine PowerPoint-Präsentation und spielt diese über \Gls{NDI} aus.
		Dieses \Gls{NDI}-Signal lässt sich wiederrum in \Gls{OBS} einbinden.
		Dies hat den Vorteil, dass kein Extra-Bildschirm freigehalten werden muss, auf dem PowerPoint läuft und abgefilmt wird.

		Die Shortcuts, um die Folien zu wechseln, sind auf dem Streamdeck programmiert.
	\section{Youlean-Loudness-Meter}
		\textit{Youlean-Loudness-Meter} ist ein \Gls{VST-Plugin}, welches die \Gls{Lautheit} eines Audiosignals plottet
	\section{PTZOptics-Control Pane}
		\textit{PTZOptics-Control Pane} ist die hauseigene Software von PTZOptics, um deren \Glsplural{PTZ-Kamera} zu steuern.
		Es können sowohl Pan, Tilt und Zoom, als auch tiefergehende Funktionen wie Weißabgleich, Belichtung oder das Menü gesteuert werden.