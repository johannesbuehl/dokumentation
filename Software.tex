\chapter{Software}
	\section{OBS}
		Über \Gls{OBS} werden die verschiedenen Videosignale gemischt, aufgenommen und an YouTube geschickt.

		\subsection{Grundfunktionen}
			In \Gls{OBS} werden verschiedene Bild-Kompositionen in sogenannten Szenen gespeichert.
			In einer Szene kann eine vielzahl von verschiedenen Quellen (Sources) verwendet werden, unter anderem:
			\begin{itemize}
				\item reine Audioeingänge
				\item Bildschirmaufnahmen
				\item Fensteraufnahmen
				\item Bilder
				\item Medienquellen (Video- oder Musikdateien)
				\item NDI (durch \nameref{ndi})
				\item andere Szenen
				\item Text
				\item Webcams
				\item Blackmagic Decklink Videoeingänge
			\end{itemize}
			Indem man sich in \Gls{OBS} mit den YouTube-Zugangsdaten anmeldet, kann man in \Gls{OBS} entweder Livestreams erstellen oder einen bereits geplanten Livestream auswählen und diesen streamen.
		\subsection{Plugins}
			\Gls{OBS} kann durch verschiedene Plugins in seiner Funktion erweitert werden.
			Installiert sind:
			\begin{itemize}
				\item obs-websocket
				\item Advanced Scene Switcher
				\item obs-ndi
				\item Audio-Monitor
			\end{itemize}

			\subsubsection{obs-websocket}
				\textit{obs-websocket} ermöglicht die Fernsteuerung von OBS durch einen Websocket.
				Dieser wird zum Beispiel von Companion genutzt, um Szenen zu wechseln oder Informationen über den aktuellen Zustand von OBS zu erlangen.
			\subsubsection{Advanced Scene Switcher}
				Mit \textit{Advanced Scene Switcher} können Abläufe in \Gls{OBS} automatisiert werden.
				Beispielsweise kann am Ende einer Medienquelle automatisch die Szene gewechselt werden oder Programme beim Schließen von \Gls{OBS} ausgeführt werden.
			\subsubsection{obs-ndi}\label{ndi}
				\textit{obs-ndi} ermöglicht die Nutzung von \Gls{NDI}-Signalen in \Gls{OBS}.
				Es fügt eine \Gls{NDI}-Source hinzu und ermöglicht eine \Gls{NDI}-Ausgabe einzelner Szenen oder des Programms.
			\subsubsection{Audio-Monitor}
				\textit{Audio-Monitor} fügt genau steuerbare Audioabhörmöglichkeiten hinzu.
				Die Standardmöglichkeiten von OBS sind in einem Untermenü versteckt und müssen erneut aktiviert werden, wenn der Kopfhörer neu eingesteckt wurde.

		\subsection{Konfiguration}
			
	\section{PPT NDI}
		Alle Text-Einblendungen werden mit PowerPoint erstellt.
		Eine Vorlage befindet sich im Google-Drive Verzeichnis.

		PPT NDI läd eine PowerPoint-Präsentation und spielt diese über \Gls{NDI} aus
		Dieses \Gls{NDI}-Signal lässt sich wiederrum in \Gls{OBS} einbinden.
		Dies hat den Vorteil, dass kein Extra-Bildschirm freigehalten werden muss, auf dem PowerPoint läuft und abgefilmt wird.

		Die Shortcuts, um die Folien zu wechseln, sind auf dem Streamdeck programmiert.
	\section{Youlean-Loudness-Meter}
		\textit{Youlean-Loudness-Meter} ist ein \Gls{VST-Plugin}, welches die \Gls{Lautheit} eines Audiosignals plottet
	\section{PTZOptics-Control Pane}
		\textit{PTZOptics-Control Pane} ist die hauseigene Software von PTZOptics, um deren \Glsplural{PTZ-Kamera} zu steuern.
		Es können sowohl Pan, Tilt und Zoom, als auch tiefergehende Funktionen wie Weißabgleich, Belichtung oder das Menü gesteuert werden.